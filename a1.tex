% Options for packages loaded elsewhere
\PassOptionsToPackage{unicode}{hyperref}
\PassOptionsToPackage{hyphens}{url}
%
\documentclass[
]{article}
\usepackage{amsmath,amssymb}
\usepackage{lmodern}
\usepackage{iftex}
\ifPDFTeX
  \usepackage[T1]{fontenc}
  \usepackage[utf8]{inputenc}
  \usepackage{textcomp} % provide euro and other symbols
\else % if luatex or xetex
  \usepackage{unicode-math}
  \defaultfontfeatures{Scale=MatchLowercase}
  \defaultfontfeatures[\rmfamily]{Ligatures=TeX,Scale=1}
\fi
% Use upquote if available, for straight quotes in verbatim environments
\IfFileExists{upquote.sty}{\usepackage{upquote}}{}
\IfFileExists{microtype.sty}{% use microtype if available
  \usepackage[]{microtype}
  \UseMicrotypeSet[protrusion]{basicmath} % disable protrusion for tt fonts
}{}
\makeatletter
\@ifundefined{KOMAClassName}{% if non-KOMA class
  \IfFileExists{parskip.sty}{%
    \usepackage{parskip}
  }{% else
    \setlength{\parindent}{0pt}
    \setlength{\parskip}{6pt plus 2pt minus 1pt}}
}{% if KOMA class
  \KOMAoptions{parskip=half}}
\makeatother
\usepackage{xcolor}
\IfFileExists{xurl.sty}{\usepackage{xurl}}{} % add URL line breaks if available
\IfFileExists{bookmark.sty}{\usepackage{bookmark}}{\usepackage{hyperref}}
\hypersetup{
  pdftitle={STAD80: Assignment 1},
  pdfauthor={Vladislav Trukhin},
  hidelinks,
  pdfcreator={LaTeX via pandoc}}
\urlstyle{same} % disable monospaced font for URLs
\usepackage[margin=1in]{geometry}
\usepackage{color}
\usepackage{fancyvrb}
\newcommand{\VerbBar}{|}
\newcommand{\VERB}{\Verb[commandchars=\\\{\}]}
\DefineVerbatimEnvironment{Highlighting}{Verbatim}{commandchars=\\\{\}}
% Add ',fontsize=\small' for more characters per line
\usepackage{framed}
\definecolor{shadecolor}{RGB}{248,248,248}
\newenvironment{Shaded}{\begin{snugshade}}{\end{snugshade}}
\newcommand{\AlertTok}[1]{\textcolor[rgb]{0.94,0.16,0.16}{#1}}
\newcommand{\AnnotationTok}[1]{\textcolor[rgb]{0.56,0.35,0.01}{\textbf{\textit{#1}}}}
\newcommand{\AttributeTok}[1]{\textcolor[rgb]{0.77,0.63,0.00}{#1}}
\newcommand{\BaseNTok}[1]{\textcolor[rgb]{0.00,0.00,0.81}{#1}}
\newcommand{\BuiltInTok}[1]{#1}
\newcommand{\CharTok}[1]{\textcolor[rgb]{0.31,0.60,0.02}{#1}}
\newcommand{\CommentTok}[1]{\textcolor[rgb]{0.56,0.35,0.01}{\textit{#1}}}
\newcommand{\CommentVarTok}[1]{\textcolor[rgb]{0.56,0.35,0.01}{\textbf{\textit{#1}}}}
\newcommand{\ConstantTok}[1]{\textcolor[rgb]{0.00,0.00,0.00}{#1}}
\newcommand{\ControlFlowTok}[1]{\textcolor[rgb]{0.13,0.29,0.53}{\textbf{#1}}}
\newcommand{\DataTypeTok}[1]{\textcolor[rgb]{0.13,0.29,0.53}{#1}}
\newcommand{\DecValTok}[1]{\textcolor[rgb]{0.00,0.00,0.81}{#1}}
\newcommand{\DocumentationTok}[1]{\textcolor[rgb]{0.56,0.35,0.01}{\textbf{\textit{#1}}}}
\newcommand{\ErrorTok}[1]{\textcolor[rgb]{0.64,0.00,0.00}{\textbf{#1}}}
\newcommand{\ExtensionTok}[1]{#1}
\newcommand{\FloatTok}[1]{\textcolor[rgb]{0.00,0.00,0.81}{#1}}
\newcommand{\FunctionTok}[1]{\textcolor[rgb]{0.00,0.00,0.00}{#1}}
\newcommand{\ImportTok}[1]{#1}
\newcommand{\InformationTok}[1]{\textcolor[rgb]{0.56,0.35,0.01}{\textbf{\textit{#1}}}}
\newcommand{\KeywordTok}[1]{\textcolor[rgb]{0.13,0.29,0.53}{\textbf{#1}}}
\newcommand{\NormalTok}[1]{#1}
\newcommand{\OperatorTok}[1]{\textcolor[rgb]{0.81,0.36,0.00}{\textbf{#1}}}
\newcommand{\OtherTok}[1]{\textcolor[rgb]{0.56,0.35,0.01}{#1}}
\newcommand{\PreprocessorTok}[1]{\textcolor[rgb]{0.56,0.35,0.01}{\textit{#1}}}
\newcommand{\RegionMarkerTok}[1]{#1}
\newcommand{\SpecialCharTok}[1]{\textcolor[rgb]{0.00,0.00,0.00}{#1}}
\newcommand{\SpecialStringTok}[1]{\textcolor[rgb]{0.31,0.60,0.02}{#1}}
\newcommand{\StringTok}[1]{\textcolor[rgb]{0.31,0.60,0.02}{#1}}
\newcommand{\VariableTok}[1]{\textcolor[rgb]{0.00,0.00,0.00}{#1}}
\newcommand{\VerbatimStringTok}[1]{\textcolor[rgb]{0.31,0.60,0.02}{#1}}
\newcommand{\WarningTok}[1]{\textcolor[rgb]{0.56,0.35,0.01}{\textbf{\textit{#1}}}}
\usepackage{graphicx}
\makeatletter
\def\maxwidth{\ifdim\Gin@nat@width>\linewidth\linewidth\else\Gin@nat@width\fi}
\def\maxheight{\ifdim\Gin@nat@height>\textheight\textheight\else\Gin@nat@height\fi}
\makeatother
% Scale images if necessary, so that they will not overflow the page
% margins by default, and it is still possible to overwrite the defaults
% using explicit options in \includegraphics[width, height, ...]{}
\setkeys{Gin}{width=\maxwidth,height=\maxheight,keepaspectratio}
% Set default figure placement to htbp
\makeatletter
\def\fps@figure{htbp}
\makeatother
\setlength{\emergencystretch}{3em} % prevent overfull lines
\providecommand{\tightlist}{%
  \setlength{\itemsep}{0pt}\setlength{\parskip}{0pt}}
\setcounter{secnumdepth}{-\maxdimen} % remove section numbering
\ifLuaTeX
  \usepackage{selnolig}  % disable illegal ligatures
\fi

\title{STAD80: Assignment 1}
\author{Vladislav Trukhin}
\date{Due: Jan 27}

\begin{document}
\maketitle

{
\setcounter{tocdepth}{2}
\tableofcontents
}
\hypertarget{question-1}{%
\subsection{Question 1}\label{question-1}}

\hypertarget{section}{%
\subsubsection{1.1}\label{section}}

B, C

\hypertarget{section-1}{%
\subsubsection{1.2}\label{section-1}}

A, B, C

\hypertarget{section-2}{%
\subsubsection{1.3}\label{section-2}}

B

\hypertarget{section-3}{%
\subsubsection{1.4}\label{section-3}}

A, C

\hypertarget{section-4}{%
\subsubsection{1.5}\label{section-4}}

A, F

\hypertarget{section-5}{%
\subsubsection{1.6}\label{section-5}}

A

\hypertarget{question-2}{%
\subsection{Question 2}\label{question-2}}

\hypertarget{section-6}{%
\subsubsection{2.1}\label{section-6}}

As
\(\sqrt{n}(\theta - \hat{\theta}_n) \overset{D}\rightarrow N(0, \frac{1}{I(\theta)})\)
and
\(\sqrt{I(\hat{\theta}_n)} \overset{P}\rightarrow \sqrt{I(\theta)}\), by
Slutsky's Theorem:

\(\sqrt{I(\hat{\theta}_n)}\sqrt{n}(\theta - \hat{\theta}_n) = \sqrt{nI(\hat{\theta}_n)}(\theta - \hat{\theta}_n) \overset{D}\rightarrow \sqrt{I(\theta)} N(0, \frac{1}{I(\theta)}) = N(0, 1)\)

Using this result:

\(lim_{n \rightarrow \infty} P(\theta \in C_n)\)

\(= lim_{n \rightarrow \infty} P(\hat{\theta}_n - \frac{z_{\alpha/2}}{\sqrt{nI(\hat{\theta}_n)}} \leq \theta \leq \hat{\theta}_n + \frac{z_{\alpha/2}}{\sqrt{nI(\hat{\theta}_n)}})\)

\(= lim_{n \rightarrow \infty} P(- z_{\alpha/2} \leq \sqrt{nI(\hat{\theta}_n)}(\theta - \hat{\theta}_n) \leq z_{\alpha/2})\)

\(= P(- z_{\alpha/2} \leq Y \leq z_{\alpha/2})\) where \(Y \sim N(0,1)\)

\(= P(Y \leq z_{\alpha/2}) - P(Y \leq -z_{\alpha/2})\)

\(= 1 - P(Y \leq -z_{\alpha/2}) - P(Y \leq -z_{\alpha/2}))\)

\(= 1 - 2P(Y \leq -z_{\alpha/2})\)

\(= 1 - 2\alpha/2\)

\(= 1 - \alpha\)

\hypertarget{a}{%
\subsubsection{2.2a}\label{a}}

\(\ell(\theta, X_1, ..., X_n) = ln(\Pi(\theta-1)x_i^{-\theta}1(x_i \geq 1))\)

\(= \Sigma(\ln((\theta-1)x_i^{-\theta}1(x_i \geq 1)))\)

\(= \Sigma(\ln(\theta-1) + \ln(x_i^{-\theta}) + \ln(1(x_i \geq 1)))\)

\(= \Sigma(\ln(\theta-1) - \theta \ln(x_i) + \ln(1(x_i \geq 1)))\)

\(= n \ln(\theta-1) - \theta \Sigma(\ln(x_i)) + \Sigma(\ln(1(x_i \geq 1))))\)

\(\frac{\partial}{\partial \theta} \ell(\theta, X_1, ..., X_n) = \frac{\partial}{\partial \theta} (n \ln(\theta-1) - \theta \Sigma(\ln(x_i)) + \Sigma(\ln(1(x_i \geq 1))))\)

\(= \frac{\partial}{\partial \theta} n \ln(\theta-1) - \frac{\partial}{\partial \theta} \theta \Sigma(\ln(x_i)) + \frac{\partial}{\partial \theta} \Sigma(\ln(1(x_i \geq 1)))\)

\(= \frac{n}{\theta-1} - \Sigma(\ln(x_i)) = 0\)

\(\Longrightarrow \frac{n}{\theta-1} = \Sigma(\ln(x_i))\)

\(\Longrightarrow n = (\theta-1)\Sigma(\ln(x_i))\)

\(\Longrightarrow n = \theta\Sigma(\ln(x_i)) - \Sigma(\ln(x_i))\)

\(\Longrightarrow n + \Sigma(\ln(x_i)) = \theta\Sigma(\ln(x_i))\)

\(\Longrightarrow \frac{n + \Sigma(\ln(x_i))}{\Sigma(\ln(x_i))} = \theta = \hat{\theta}_n\)

\hypertarget{b}{%
\subsubsection{2.2b}\label{b}}

\(I(\theta) = E(-\frac{\partial^2}{\partial \theta^2} \ln p_{\theta}(X))\)

\(= E(-\frac{\partial^2}{\partial \theta^2} (\ln((\theta-1)X^{-\theta}1(X \geq 1))))\)

\(= E(-\frac{\partial^2}{\partial \theta^2} (\ln(\theta-1) + \ln(X^{-\theta}) + \ln(1(X \geq 1))))\)

\(= E(-\frac{\partial^2}{\partial \theta^2} (\ln(\theta-1) - \theta \ln(X) + \ln(1(X \geq 1))))\)

\(= E(-\frac{\partial^2}{\partial \theta^2} \ln(\theta-1) + \frac{\partial^2}{\partial \theta^2}\theta \ln(X) - \frac{\partial^2}{\partial \theta^2}\ln(1(X \geq 1)))\)

\(= E(-\frac{\partial}{\partial \theta} \frac{1}{\theta-1})\)

\(= E(\frac{1}{(\theta-1)^2})\)

\(=\int_{-\infty}^{\infty} \frac{1}{(\theta-1)^2} (\theta-1)X^{-\theta}1(X \geq 1)\)

\(=\int_{1}^{\infty} \frac{1}{(\theta-1)}X^{-\theta}\)

\(=\frac{1}{(\theta-1)} \int_{1}^{\infty} X^{-\theta}\)

\(=\frac{1}{(\theta-1)} \frac{X^{-\theta+1}}{-\theta+1}\vert_{1}^{\infty}\)

\(= -\frac{1}{(\theta-1)^2} X^{-\theta+1}\vert_{1}^{\infty}\)

\(= -\frac{1}{(\theta-1)^2} * 0 + \frac{1}{(\theta-1)^2} * 1\)

\(= \frac{1}{(\theta-1)^2}\)

\(\Longrightarrow \frac{1}{I(\theta)} = (\theta-1)^2\)

\hypertarget{c}{%
\subsubsection{2.2c}\label{c}}

\(C_n = [\hat{\theta}_n - \frac{z_{\alpha/2}}{\sqrt{nI(\hat{\theta}_n)}}, \hat{\theta}_n + \frac{z_{\alpha/2}}{\sqrt{nI(\hat{\theta}_n)}}]\)

\(= [\hat{\theta}_n - \frac{z_{\alpha/2}}{\sqrt{\frac{n}{(\hat{\theta}_n-1)^2}}}, \hat{\theta}_n + \frac{z_{\alpha/2}}{\sqrt{\frac{n}{(\hat{\theta}_n-1)^2}}}]\)

\(= [\hat{\theta}_n - \frac{z_{\alpha/2}}{\frac{\sqrt{n}}{(\hat{\theta}_n-1)}}, \hat{\theta}_n + \frac{z_{\alpha/2}}{\frac{\sqrt{n}}{(\hat{\theta}_n-1)}}]\)

\(= [\hat{\theta}_n - z_{\alpha/2}\frac{(\hat{\theta}_n-1)}{\sqrt{n}}, \hat{\theta}_n + z_{\alpha/2}\frac{(\hat{\theta}_n-1)}{\sqrt{n}}]\)

\(= [\hat{\theta}_n - 1.96\frac{(\hat{\theta}_n-1)}{\sqrt{n}}, \hat{\theta}_n + 1.96\frac{(\hat{\theta}_n-1)}{\sqrt{n}}]\)

\hypertarget{d}{%
\subsubsection{2.2d}\label{d}}

\begin{Shaded}
\begin{Highlighting}[]
\NormalTok{invcdf }\OtherTok{\textless{}{-}} \ControlFlowTok{function}\NormalTok{(y, theta) \{}
  \FunctionTok{return}\NormalTok{ ((}\DecValTok{1}\SpecialCharTok{{-}}\NormalTok{y)}\SpecialCharTok{\^{}}\NormalTok{(}\DecValTok{1}\SpecialCharTok{/}\NormalTok{(}\SpecialCharTok{{-}}\NormalTok{theta}\SpecialCharTok{+}\DecValTok{1}\NormalTok{)))}
\NormalTok{\}}

\NormalTok{N}\OtherTok{=}\DecValTok{10000}
\NormalTok{n}\OtherTok{=}\DecValTok{100}
\NormalTok{theta}\OtherTok{=}\DecValTok{2}
\NormalTok{count}\OtherTok{=}\DecValTok{0}
\ControlFlowTok{for}\NormalTok{ (i }\ControlFlowTok{in} \DecValTok{1}\SpecialCharTok{:}\NormalTok{N) \{}
\NormalTok{  Y }\OtherTok{\textless{}{-}} \FunctionTok{runif}\NormalTok{(n, }\DecValTok{0}\NormalTok{, }\DecValTok{1}\NormalTok{)}
\NormalTok{  X }\OtherTok{\textless{}{-}} \FunctionTok{invcdf}\NormalTok{(Y, }\DecValTok{2}\NormalTok{)}
\NormalTok{  theta\_hat }\OtherTok{\textless{}{-}}\NormalTok{ (n }\SpecialCharTok{+} \FunctionTok{sum}\NormalTok{(}\FunctionTok{log}\NormalTok{(X))) }\SpecialCharTok{/} \FunctionTok{sum}\NormalTok{(}\FunctionTok{log}\NormalTok{(X))}
\NormalTok{  c\_l }\OtherTok{\textless{}{-}}\NormalTok{ theta\_hat }\SpecialCharTok{{-}} \FloatTok{1.96}\SpecialCharTok{*}\NormalTok{(theta\_hat}\DecValTok{{-}1}\NormalTok{)}\SpecialCharTok{/}\FunctionTok{sqrt}\NormalTok{(n)}
\NormalTok{  c\_u }\OtherTok{\textless{}{-}}\NormalTok{ theta\_hat }\SpecialCharTok{+} \FloatTok{1.96}\SpecialCharTok{*}\NormalTok{(theta\_hat}\DecValTok{{-}1}\NormalTok{)}\SpecialCharTok{/}\FunctionTok{sqrt}\NormalTok{(n)}
  \ControlFlowTok{if}\NormalTok{ (c\_l }\SpecialCharTok{\textless{}=}\NormalTok{ theta }\SpecialCharTok{\&}\NormalTok{ theta }\SpecialCharTok{\textless{}=}\NormalTok{ c\_u) \{}
\NormalTok{    count }\OtherTok{=}\NormalTok{ count }\SpecialCharTok{+} \DecValTok{1}
\NormalTok{  \}}
\NormalTok{\}}
\NormalTok{count}\SpecialCharTok{/}\NormalTok{N}
\end{Highlighting}
\end{Shaded}

\begin{verbatim}
## [1] 0.9537
\end{verbatim}

Therefore the 95\% CI is effective.

\hypertarget{question-3}{%
\subsection{Question 3}\label{question-3}}

\hypertarget{a-1}{%
\subsubsection{3a}\label{a-1}}

\begin{Shaded}
\begin{Highlighting}[]
\NormalTok{generate }\OtherTok{\textless{}{-}} \ControlFlowTok{function}\NormalTok{(n)\{}
\NormalTok{  N }\OtherTok{\textless{}{-}} \DecValTok{10000}
\NormalTok{  Xbar\_n }\OtherTok{\textless{}{-}} \FunctionTok{vector}\NormalTok{(}\AttributeTok{mode =} \StringTok{"list"}\NormalTok{, }\AttributeTok{length =}\NormalTok{ N)}
  \ControlFlowTok{for}\NormalTok{ (j }\ControlFlowTok{in} \DecValTok{1}\SpecialCharTok{:}\NormalTok{N) \{}
\NormalTok{    X }\OtherTok{=} \FunctionTok{runif}\NormalTok{(n, }\DecValTok{0}\NormalTok{, }\DecValTok{1}\NormalTok{)}
    \ControlFlowTok{for}\NormalTok{ (i }\ControlFlowTok{in} \DecValTok{1}\SpecialCharTok{:}\NormalTok{n) \{}
      \ControlFlowTok{if}\NormalTok{ (X[i] }\SpecialCharTok{\textless{}} \FloatTok{0.5}\NormalTok{) \{X[i] }\OtherTok{=} \SpecialCharTok{{-}}\DecValTok{1}\NormalTok{\}}
      \ControlFlowTok{else}\NormalTok{ \{X[i] }\OtherTok{=} \DecValTok{1}\NormalTok{\}}
\NormalTok{    \}}
\NormalTok{  Xbar\_n[j] }\OtherTok{\textless{}{-}} \FunctionTok{mean}\NormalTok{(X)}
\NormalTok{  \}}
\FunctionTok{return}\NormalTok{(Xbar\_n)}
\NormalTok{\}}

\NormalTok{Xbar\_10 }\OtherTok{\textless{}{-}} \FunctionTok{as.numeric}\NormalTok{(}\FunctionTok{generate}\NormalTok{(}\DecValTok{10}\NormalTok{))}
\NormalTok{Xbar\_100 }\OtherTok{\textless{}{-}} \FunctionTok{as.numeric}\NormalTok{(}\FunctionTok{generate}\NormalTok{(}\DecValTok{100}\NormalTok{))}
\NormalTok{Xbar\_1000 }\OtherTok{\textless{}{-}} \FunctionTok{as.numeric}\NormalTok{(}\FunctionTok{generate}\NormalTok{(}\DecValTok{1000}\NormalTok{))}
\NormalTok{Xbar\_10000 }\OtherTok{\textless{}{-}} \FunctionTok{as.numeric}\NormalTok{(}\FunctionTok{generate}\NormalTok{(}\DecValTok{10000}\NormalTok{))}

\FunctionTok{curve}\NormalTok{(}\FunctionTok{log10}\NormalTok{(x), }\AttributeTok{from=}\DecValTok{1}\NormalTok{, }\AttributeTok{to=}\DecValTok{10000}\NormalTok{, }\AttributeTok{ylim=}\FunctionTok{c}\NormalTok{(}\SpecialCharTok{{-}}\DecValTok{2}\NormalTok{,}\DecValTok{4}\NormalTok{), }\AttributeTok{log=}\StringTok{"x"}\NormalTok{, }\AttributeTok{xlab =} \StringTok{"n"}\NormalTok{, }\AttributeTok{ylab =} \StringTok{"log10(n)"}\NormalTok{)}
\FunctionTok{abline}\NormalTok{(}\AttributeTok{h =} \DecValTok{0}\NormalTok{, }\AttributeTok{col=}\StringTok{"blue"}\NormalTok{)}
\FunctionTok{points}\NormalTok{(}\DecValTok{10}\NormalTok{, Xbar\_10[}\DecValTok{1}\NormalTok{] }\SpecialCharTok{{-}} \DecValTok{0}\NormalTok{)}
\FunctionTok{points}\NormalTok{(}\DecValTok{100}\NormalTok{, Xbar\_100[}\DecValTok{1}\NormalTok{] }\SpecialCharTok{{-}} \DecValTok{0}\NormalTok{)}
\FunctionTok{points}\NormalTok{(}\DecValTok{1000}\NormalTok{, Xbar\_1000[}\DecValTok{1}\NormalTok{] }\SpecialCharTok{{-}} \DecValTok{0}\NormalTok{)}
\FunctionTok{points}\NormalTok{(}\DecValTok{10000}\NormalTok{, Xbar\_10000[}\DecValTok{1}\NormalTok{] }\SpecialCharTok{{-}} \DecValTok{0}\NormalTok{)}
\end{Highlighting}
\end{Shaded}

\includegraphics{assignment1_vladislav_trukhin_files/figure-latex/unnamed-chunk-2-1.pdf}

The plot shows as \(n \rightarrow \infty\),
\((\bar{X}_n^{(1)} - \mu) \rightarrow 0\) or
\(\bar{X}_n^{(1)} \rightarrow \mu\).

\hypertarget{b-1}{%
\subsubsection{3b}\label{b-1}}

\begin{Shaded}
\begin{Highlighting}[]
\NormalTok{lln }\OtherTok{\textless{}{-}} \ControlFlowTok{function}\NormalTok{(X, e) \{}
\NormalTok{  N }\OtherTok{\textless{}{-}} \DecValTok{10000}
  \ControlFlowTok{for}\NormalTok{ (i }\ControlFlowTok{in} \DecValTok{1}\SpecialCharTok{:}\NormalTok{N) \{}
    \ControlFlowTok{if}\NormalTok{ (}\FunctionTok{abs}\NormalTok{(X[i] }\SpecialCharTok{{-}} \DecValTok{0}\NormalTok{) }\SpecialCharTok{\textgreater{}}\NormalTok{ e) \{X[i] }\OtherTok{=} \DecValTok{1}\NormalTok{\}}
    \ControlFlowTok{else}\NormalTok{ \{X[i] }\OtherTok{=} \DecValTok{0}\NormalTok{\}}
\NormalTok{  \}}
  \FunctionTok{return}\NormalTok{(}\FunctionTok{mean}\NormalTok{(X))}
\NormalTok{\}}

\FunctionTok{curve}\NormalTok{(}\FunctionTok{log10}\NormalTok{(x), }\AttributeTok{from=}\DecValTok{1}\NormalTok{, }\AttributeTok{to=}\DecValTok{10000}\NormalTok{, }\AttributeTok{ylim=}\FunctionTok{c}\NormalTok{(}\SpecialCharTok{{-}}\DecValTok{2}\NormalTok{,}\DecValTok{4}\NormalTok{), }\AttributeTok{log=}\StringTok{"x"}\NormalTok{, }\AttributeTok{xlab =} \StringTok{"n"}\NormalTok{, }\AttributeTok{ylab =} \StringTok{"log10(n)"}\NormalTok{)}
\FunctionTok{abline}\NormalTok{(}\AttributeTok{h =} \DecValTok{0}\NormalTok{, }\AttributeTok{col=}\StringTok{"blue"}\NormalTok{)}
\FunctionTok{points}\NormalTok{(}\DecValTok{10}\NormalTok{, }\FunctionTok{lln}\NormalTok{(Xbar\_10, }\FloatTok{0.5}\NormalTok{) }\SpecialCharTok{{-}} \DecValTok{0}\NormalTok{)}
\FunctionTok{points}\NormalTok{(}\DecValTok{100}\NormalTok{, }\FunctionTok{lln}\NormalTok{(Xbar\_100, }\FloatTok{0.5}\NormalTok{) }\SpecialCharTok{{-}} \DecValTok{0}\NormalTok{)}
\FunctionTok{points}\NormalTok{(}\DecValTok{1000}\NormalTok{, }\FunctionTok{lln}\NormalTok{(Xbar\_1000, }\FloatTok{0.5}\NormalTok{) }\SpecialCharTok{{-}} \DecValTok{0}\NormalTok{)}
\FunctionTok{points}\NormalTok{(}\DecValTok{10000}\NormalTok{, }\FunctionTok{lln}\NormalTok{(Xbar\_10000, }\FloatTok{0.5}\NormalTok{) }\SpecialCharTok{{-}} \DecValTok{0}\NormalTok{)}

\FunctionTok{points}\NormalTok{(}\DecValTok{10}\NormalTok{, }\FunctionTok{lln}\NormalTok{(Xbar\_10, }\FloatTok{0.1}\NormalTok{) }\SpecialCharTok{{-}} \DecValTok{0}\NormalTok{, }\AttributeTok{col =} \StringTok{"red"}\NormalTok{)}
\FunctionTok{points}\NormalTok{(}\DecValTok{100}\NormalTok{, }\FunctionTok{lln}\NormalTok{(Xbar\_100, }\FloatTok{0.1}\NormalTok{) }\SpecialCharTok{{-}} \DecValTok{0}\NormalTok{, }\AttributeTok{col =} \StringTok{"red"}\NormalTok{)}
\FunctionTok{points}\NormalTok{(}\DecValTok{1000}\NormalTok{, }\FunctionTok{lln}\NormalTok{(Xbar\_1000, }\FloatTok{0.1}\NormalTok{) }\SpecialCharTok{{-}} \DecValTok{0}\NormalTok{, }\AttributeTok{col =} \StringTok{"red"}\NormalTok{)}
\FunctionTok{points}\NormalTok{(}\DecValTok{10000}\NormalTok{, }\FunctionTok{lln}\NormalTok{(Xbar\_10000, }\FloatTok{0.1}\NormalTok{) }\SpecialCharTok{{-}} \DecValTok{0}\NormalTok{, }\AttributeTok{col =} \StringTok{"red"}\NormalTok{)}

\FunctionTok{points}\NormalTok{(}\DecValTok{10}\NormalTok{, }\FunctionTok{lln}\NormalTok{(Xbar\_10, }\FloatTok{0.05}\NormalTok{) }\SpecialCharTok{{-}} \DecValTok{0}\NormalTok{, }\AttributeTok{col =} \StringTok{"yellow"}\NormalTok{)}
\FunctionTok{points}\NormalTok{(}\DecValTok{100}\NormalTok{, }\FunctionTok{lln}\NormalTok{(Xbar\_100, }\FloatTok{0.05}\NormalTok{) }\SpecialCharTok{{-}} \DecValTok{0}\NormalTok{, }\AttributeTok{col =} \StringTok{"yellow"}\NormalTok{)}
\FunctionTok{points}\NormalTok{(}\DecValTok{1000}\NormalTok{, }\FunctionTok{lln}\NormalTok{(Xbar\_1000, }\FloatTok{0.05}\NormalTok{) }\SpecialCharTok{{-}} \DecValTok{0}\NormalTok{, }\AttributeTok{col =} \StringTok{"yellow"}\NormalTok{)}
\FunctionTok{points}\NormalTok{(}\DecValTok{10000}\NormalTok{, }\FunctionTok{lln}\NormalTok{(Xbar\_10000, }\FloatTok{0.05}\NormalTok{) }\SpecialCharTok{{-}} \DecValTok{0}\NormalTok{, }\AttributeTok{col =} \StringTok{"yellow"}\NormalTok{)}
\end{Highlighting}
\end{Shaded}

\includegraphics{assignment1_vladislav_trukhin_files/figure-latex/unnamed-chunk-3-1.pdf}

The plot shows that
\(\underset{n \rightarrow \infty}{lim}P(|\bar{X}_n^{(i)} - \mu| > \epsilon) = 0\)
\(\forall \epsilon\) \(\forall i\), which illustrates the Law of Large
Numbers, or \(\bar{X}_n^{(i)} \overset{P}{\rightarrow} \mu\).

\hypertarget{c-1}{%
\subsubsection{3c}\label{c-1}}

\begin{Shaded}
\begin{Highlighting}[]
\FunctionTok{par}\NormalTok{(}\AttributeTok{mfrow=}\FunctionTok{c}\NormalTok{(}\DecValTok{1}\NormalTok{,}\DecValTok{2}\NormalTok{))}
\FunctionTok{hist}\NormalTok{(}\FunctionTok{sqrt}\NormalTok{(}\DecValTok{10}\NormalTok{)}\SpecialCharTok{*}\NormalTok{(Xbar\_10 }\SpecialCharTok{{-}} \DecValTok{0}\NormalTok{)}\SpecialCharTok{/}\DecValTok{1}\NormalTok{, }\AttributeTok{main =} \StringTok{"Histogram n=10"}\NormalTok{, }\AttributeTok{xlab =} \StringTok{"root(n)*(Xbar{-}mu)/sigma"}\NormalTok{)}
\FunctionTok{qqnorm}\NormalTok{(}\FunctionTok{sqrt}\NormalTok{(}\DecValTok{10}\NormalTok{)}\SpecialCharTok{*}\NormalTok{(Xbar\_10 }\SpecialCharTok{{-}} \DecValTok{0}\NormalTok{)}\SpecialCharTok{/}\DecValTok{1}\NormalTok{, }\AttributeTok{main =} \StringTok{"QQ Plot n=10"}\NormalTok{)}
\end{Highlighting}
\end{Shaded}

\includegraphics{assignment1_vladislav_trukhin_files/figure-latex/unnamed-chunk-4-1.pdf}

\begin{Shaded}
\begin{Highlighting}[]
\FunctionTok{hist}\NormalTok{(}\FunctionTok{sqrt}\NormalTok{(}\DecValTok{1000}\NormalTok{)}\SpecialCharTok{*}\NormalTok{(Xbar\_1000 }\SpecialCharTok{{-}} \DecValTok{0}\NormalTok{)}\SpecialCharTok{/}\DecValTok{1}\NormalTok{, }\AttributeTok{main =} \StringTok{"Histogram n=1000"}\NormalTok{, }\AttributeTok{xlab =} \StringTok{"root(n)*(Xbar{-}mu)/sigma"}\NormalTok{)}
\FunctionTok{qqnorm}\NormalTok{(}\FunctionTok{sqrt}\NormalTok{(}\DecValTok{1000}\NormalTok{)}\SpecialCharTok{*}\NormalTok{(Xbar\_1000 }\SpecialCharTok{{-}} \DecValTok{0}\NormalTok{)}\SpecialCharTok{/}\DecValTok{1}\NormalTok{, }\AttributeTok{main =} \StringTok{"QQ Plot n=1000"}\NormalTok{)}
\end{Highlighting}
\end{Shaded}

\includegraphics{assignment1_vladislav_trukhin_files/figure-latex/unnamed-chunk-4-2.pdf}

\begin{Shaded}
\begin{Highlighting}[]
\FunctionTok{hist}\NormalTok{(}\FunctionTok{sqrt}\NormalTok{(}\DecValTok{10000}\NormalTok{)}\SpecialCharTok{*}\NormalTok{(Xbar\_10000 }\SpecialCharTok{{-}} \DecValTok{0}\NormalTok{)}\SpecialCharTok{/}\DecValTok{1}\NormalTok{, }\AttributeTok{main =} \StringTok{"Histogram n=10000"}\NormalTok{, }\AttributeTok{xlab =} \StringTok{"root(n)*(Xbar{-}mu)/sigma"}\NormalTok{)}
\FunctionTok{qqnorm}\NormalTok{(}\FunctionTok{sqrt}\NormalTok{(}\DecValTok{10000}\NormalTok{)}\SpecialCharTok{*}\NormalTok{(Xbar\_10000 }\SpecialCharTok{{-}} \DecValTok{0}\NormalTok{)}\SpecialCharTok{/}\DecValTok{1}\NormalTok{, }\AttributeTok{main =} \StringTok{"QQ Plot n=10000"}\NormalTok{)}
\end{Highlighting}
\end{Shaded}

\includegraphics{assignment1_vladislav_trukhin_files/figure-latex/unnamed-chunk-4-3.pdf}

As \(n \rightarrow \infty\), the histogram of
\(\sqrt{n}(\bar{X}_n^{(i)} - \mu)/\sigma\) begins to more closely
resemble a random sample generated by a standard Normal distribution.
The QQ plot begins to more closely resemble the \(x = y\) line,
suggesting the data follows a standard Normal distribution. This
illustrates the Central Limit Theorem, where
\(\sqrt{n}(\bar{X}_n^{(i)} - \mu)/\sigma \overset{D}{\rightarrow} N(0, 1)\).

\hypertarget{d-1}{%
\subsubsection{3.d}\label{d-1}}

\begin{Shaded}
\begin{Highlighting}[]
\NormalTok{conv\_prob }\OtherTok{\textless{}{-}} \ControlFlowTok{function}\NormalTok{(X, e) \{}
  \ControlFlowTok{for}\NormalTok{ (i }\ControlFlowTok{in} \DecValTok{1}\SpecialCharTok{:}\DecValTok{10000}\NormalTok{) \{}
    \ControlFlowTok{if}\NormalTok{(}\FunctionTok{abs}\NormalTok{(X[i] }\SpecialCharTok{{-}} \FunctionTok{rnorm}\NormalTok{(}\DecValTok{1}\NormalTok{)) }\SpecialCharTok{\textgreater{}}\NormalTok{ e) \{X[i] }\OtherTok{=} \DecValTok{1}\NormalTok{\}}
    \ControlFlowTok{else}\NormalTok{ \{X[i] }\OtherTok{=} \DecValTok{0}\NormalTok{\}}
\NormalTok{  \}}
\FunctionTok{return}\NormalTok{(}\FunctionTok{mean}\NormalTok{(X))}
\NormalTok{\}}

\FunctionTok{curve}\NormalTok{(}\FunctionTok{log10}\NormalTok{(x), }\AttributeTok{from=}\DecValTok{1}\NormalTok{, }\AttributeTok{to=}\DecValTok{10000}\NormalTok{, }\AttributeTok{ylim=}\FunctionTok{c}\NormalTok{(}\SpecialCharTok{{-}}\DecValTok{2}\NormalTok{,}\DecValTok{4}\NormalTok{), }\AttributeTok{log=}\StringTok{"x"}\NormalTok{, }\AttributeTok{xlab =} \StringTok{"n"}\NormalTok{, }\AttributeTok{ylab =} \StringTok{"log10(n)"}\NormalTok{)}
\FunctionTok{abline}\NormalTok{(}\AttributeTok{h =} \DecValTok{0}\NormalTok{, }\AttributeTok{col=}\StringTok{"blue"}\NormalTok{)}
\FunctionTok{points}\NormalTok{(}\DecValTok{10}\NormalTok{, }\FunctionTok{conv\_prob}\NormalTok{(}\FunctionTok{sqrt}\NormalTok{(}\DecValTok{10}\NormalTok{)}\SpecialCharTok{*}\NormalTok{(Xbar\_10 }\SpecialCharTok{{-}} \DecValTok{0}\NormalTok{)}\SpecialCharTok{/}\DecValTok{1}\NormalTok{, }\FloatTok{0.001}\NormalTok{))}
\FunctionTok{points}\NormalTok{(}\DecValTok{100}\NormalTok{, }\FunctionTok{conv\_prob}\NormalTok{(}\FunctionTok{sqrt}\NormalTok{(}\DecValTok{100}\NormalTok{)}\SpecialCharTok{*}\NormalTok{(Xbar\_100 }\SpecialCharTok{{-}} \DecValTok{0}\NormalTok{)}\SpecialCharTok{/}\DecValTok{1}\NormalTok{, }\FloatTok{0.001}\NormalTok{))}
\FunctionTok{points}\NormalTok{(}\DecValTok{1000}\NormalTok{, }\FunctionTok{conv\_prob}\NormalTok{(}\FunctionTok{sqrt}\NormalTok{(}\DecValTok{1000}\NormalTok{)}\SpecialCharTok{*}\NormalTok{(Xbar\_1000 }\SpecialCharTok{{-}} \DecValTok{0}\NormalTok{)}\SpecialCharTok{/}\DecValTok{1}\NormalTok{, }\FloatTok{0.001}\NormalTok{))}
\FunctionTok{points}\NormalTok{(}\DecValTok{10000}\NormalTok{, }\FunctionTok{conv\_prob}\NormalTok{(}\FunctionTok{sqrt}\NormalTok{(}\DecValTok{10000}\NormalTok{)}\SpecialCharTok{*}\NormalTok{(Xbar\_10000 }\SpecialCharTok{{-}} \DecValTok{0}\NormalTok{)}\SpecialCharTok{/}\DecValTok{1}\NormalTok{, }\FloatTok{0.001}\NormalTok{))}
\end{Highlighting}
\end{Shaded}

\includegraphics{assignment1_vladislav_trukhin_files/figure-latex/unnamed-chunk-5-1.pdf}

The plot shows that
\(\underset{n \rightarrow \infty}{lim}P(|\sqrt{n}(\bar{X}_n^{(i)} - \mu)/\sigma - Y_i| > \epsilon) = 1\)
for \(\epsilon = 0.001\) \(\forall i\), which implies that
\(\sqrt{n}(\bar{X}_n^{(i)} - \mu)/\sigma\) does not converge in
probability to \(Y_i\).

\hypertarget{question-4}{%
\subsection{Question 4}\label{question-4}}

\hypertarget{a-2}{%
\subsubsection{4a}\label{a-2}}

\begin{Shaded}
\begin{Highlighting}[]
\NormalTok{X }\OtherTok{\textless{}{-}} \FunctionTok{read.table}\NormalTok{(}\StringTok{"/Users/vladislavtrukhin/Downloads/datasets\_all/ratings.dat"}\NormalTok{, }
                \AttributeTok{sep=} \StringTok{","}\NormalTok{)}
\FunctionTok{names}\NormalTok{(X) }\OtherTok{\textless{}{-}} \FunctionTok{c}\NormalTok{(}\StringTok{"UserID"}\NormalTok{, }\StringTok{"ProfileID"}\NormalTok{, }\StringTok{"Rating"}\NormalTok{)}

\CommentTok{\#Function to calculate weighted rank}
\NormalTok{weighted.rank }\OtherTok{\textless{}{-}} \ControlFlowTok{function}\NormalTok{(ProfileID) \{}
\NormalTok{  R }\OtherTok{\textless{}{-}} \FunctionTok{mean}\NormalTok{(X[}\FunctionTok{which}\NormalTok{(X}\SpecialCharTok{$}\NormalTok{ProfileID }\SpecialCharTok{==}\NormalTok{ ProfileID), }\StringTok{\textquotesingle{}Rating\textquotesingle{}}\NormalTok{])}
\NormalTok{  v }\OtherTok{\textless{}{-}} \FunctionTok{nrow}\NormalTok{(X[}\FunctionTok{which}\NormalTok{(X}\SpecialCharTok{$}\NormalTok{ProfileID }\SpecialCharTok{==}\NormalTok{ ProfileID), ])}
\NormalTok{  m }\OtherTok{\textless{}{-}} \DecValTok{4182}
\NormalTok{  C }\OtherTok{\textless{}{-}} \FunctionTok{mean}\NormalTok{(X[, }\DecValTok{3}\NormalTok{])}
  \FunctionTok{return}\NormalTok{ ((v}\SpecialCharTok{/}\NormalTok{(v}\SpecialCharTok{+}\NormalTok{m))}\SpecialCharTok{*}\NormalTok{R }\SpecialCharTok{+}\NormalTok{ (m}\SpecialCharTok{/}\NormalTok{(v}\SpecialCharTok{+}\NormalTok{m))}\SpecialCharTok{*}\NormalTok{C)}
\NormalTok{\}}

\CommentTok{\#Histogram of weighted ranks of all ProfileIDs associated with UserID 100}
\NormalTok{results }\OtherTok{\textless{}{-}} \FunctionTok{c}\NormalTok{()}
\ControlFlowTok{for}\NormalTok{ (i }\ControlFlowTok{in}\NormalTok{ X[}\FunctionTok{which}\NormalTok{(X}\SpecialCharTok{$}\NormalTok{UserID }\SpecialCharTok{==} \DecValTok{100}\NormalTok{), }\StringTok{\textquotesingle{}ProfileID\textquotesingle{}}\NormalTok{]) \{}
\NormalTok{  results }\OtherTok{\textless{}{-}} \FunctionTok{c}\NormalTok{(results, }\FunctionTok{weighted.rank}\NormalTok{(i))}
\NormalTok{\}}
\FunctionTok{hist}\NormalTok{(results, }\AttributeTok{main =} \StringTok{"Weighted Ratings of all ProfileIDs associated with UserID 100"}\NormalTok{, }\AttributeTok{xlab=}\StringTok{"Rating"}\NormalTok{)}
\end{Highlighting}
\end{Shaded}

\includegraphics{assignment1_vladislav_trukhin_files/figure-latex/unnamed-chunk-6-1.pdf}

\hypertarget{b-2}{%
\subsubsection{4b}\label{b-2}}

\begin{Shaded}
\begin{Highlighting}[]
\FunctionTok{par}\NormalTok{(}\AttributeTok{mfrow=}\FunctionTok{c}\NormalTok{(}\DecValTok{1}\NormalTok{,}\DecValTok{2}\NormalTok{))}
\FunctionTok{load}\NormalTok{(}\StringTok{"/Users/vladislavtrukhin/Downloads/datasets\_all/users.Rdata"}\NormalTok{)}

\CommentTok{\#Female CA Users}
\NormalTok{ca }\OtherTok{\textless{}{-}} \FunctionTok{grep}\NormalTok{(}\StringTok{"\^{}(?!.*CAR).*CA"}\NormalTok{, User}\SpecialCharTok{$}\NormalTok{State, }\AttributeTok{perl=}\ConstantTok{TRUE}\NormalTok{, }\AttributeTok{ignore.case =} \ConstantTok{TRUE}\NormalTok{)}
\NormalTok{ca\_f }\OtherTok{\textless{}{-}}\NormalTok{ ca[}\FunctionTok{which}\NormalTok{(User}\SpecialCharTok{$}\NormalTok{Gender[ca] }\SpecialCharTok{==} \StringTok{"F"}\NormalTok{)]}

\CommentTok{\#Male NY Users}
\NormalTok{ny }\OtherTok{\textless{}{-}} \FunctionTok{grep}\NormalTok{(}\StringTok{".*ny|.*york"}\NormalTok{, User}\SpecialCharTok{$}\NormalTok{State, }\AttributeTok{perl=}\ConstantTok{TRUE}\NormalTok{, }\AttributeTok{ignore.case =} \ConstantTok{TRUE}\NormalTok{)}
\NormalTok{ny\_m }\OtherTok{\textless{}{-}}\NormalTok{ ny[}\FunctionTok{which}\NormalTok{(User}\SpecialCharTok{$}\NormalTok{Gender[ny] }\SpecialCharTok{==} \StringTok{"M"}\NormalTok{)]}

\CommentTok{\#Box{-}plot of ratings given out by female CA users}
\NormalTok{ca\_f\_ratings }\OtherTok{\textless{}{-}} \FunctionTok{c}\NormalTok{()}
\ControlFlowTok{for}\NormalTok{ (i }\ControlFlowTok{in}\NormalTok{ ca\_f) \{}
\NormalTok{  ca\_f\_ratings }\OtherTok{\textless{}{-}} \FunctionTok{c}\NormalTok{(ca\_f\_ratings, X[}\FunctionTok{which}\NormalTok{(X}\SpecialCharTok{$}\NormalTok{UserID }\SpecialCharTok{==}\NormalTok{ i), }\StringTok{\textquotesingle{}Rating\textquotesingle{}}\NormalTok{])}
\NormalTok{\}}
\FunctionTok{boxplot}\NormalTok{(ca\_f\_ratings, }\AttributeTok{main=}\StringTok{"Ratings from F CA Users"}\NormalTok{, }\AttributeTok{ylab=}\StringTok{"Rating"}\NormalTok{)}

\CommentTok{\#Box{-}plot of ratings given out by male NY users}
\NormalTok{ny\_m\_ratings }\OtherTok{\textless{}{-}} \FunctionTok{c}\NormalTok{() }
\ControlFlowTok{for}\NormalTok{ (i }\ControlFlowTok{in}\NormalTok{ ny\_m) \{}
\NormalTok{  ny\_m\_ratings }\OtherTok{\textless{}{-}} \FunctionTok{c}\NormalTok{(ny\_m\_ratings, X[}\FunctionTok{which}\NormalTok{(X}\SpecialCharTok{$}\NormalTok{UserID }\SpecialCharTok{==}\NormalTok{ i), }\StringTok{\textquotesingle{}Rating\textquotesingle{}}\NormalTok{])}
\NormalTok{\}}
\FunctionTok{boxplot}\NormalTok{(ny\_m\_ratings, }\AttributeTok{main=}\StringTok{"Ratings from M NY Users"}\NormalTok{, }\AttributeTok{ylab=}\StringTok{"Rating"}\NormalTok{)}
\end{Highlighting}
\end{Shaded}

\includegraphics{assignment1_vladislav_trukhin_files/figure-latex/unnamed-chunk-7-1.pdf}

\hypertarget{c-2}{%
\subsubsection{4c}\label{c-2}}

\begin{Shaded}
\begin{Highlighting}[]
\FunctionTok{library}\NormalTok{(biganalytics)}
\end{Highlighting}
\end{Shaded}

\begin{verbatim}
## Loading required package: bigmemory
\end{verbatim}

\begin{verbatim}
## Loading required package: foreach
\end{verbatim}

\begin{verbatim}
## Loading required package: biglm
\end{verbatim}

\begin{verbatim}
## Loading required package: DBI
\end{verbatim}

\begin{Shaded}
\begin{Highlighting}[]
\CommentTok{\#Given}
\NormalTok{N}\OtherTok{=}\DecValTok{3000000}       
\NormalTok{Nu}\OtherTok{=}\DecValTok{135359}
\NormalTok{Np}\OtherTok{=}\DecValTok{220970}
\NormalTok{user.rat}\OtherTok{=}\FunctionTok{rep}\NormalTok{(}\DecValTok{0}\NormalTok{,Nu)}
\NormalTok{user.num}\OtherTok{=}\FunctionTok{rep}\NormalTok{(}\DecValTok{0}\NormalTok{,Nu)}
\NormalTok{profile.rat}\OtherTok{=}\FunctionTok{rep}\NormalTok{(}\DecValTok{0}\NormalTok{,Np)}
\NormalTok{profile.num}\OtherTok{=}\FunctionTok{rep}\NormalTok{(}\DecValTok{0}\NormalTok{,Np)}
\ControlFlowTok{for}\NormalTok{ (i }\ControlFlowTok{in} \DecValTok{1}\SpecialCharTok{:}\NormalTok{N)\{}
\NormalTok{    user.rat[X[i,}\StringTok{\textquotesingle{}UserID\textquotesingle{}}\NormalTok{]]}\OtherTok{=}\NormalTok{user.rat[X[i,}\StringTok{\textquotesingle{}UserID\textquotesingle{}}\NormalTok{]]}\SpecialCharTok{+}\NormalTok{X[i,}\StringTok{\textquotesingle{}Rating\textquotesingle{}}\NormalTok{]}
\NormalTok{    user.num[X[i,}\StringTok{\textquotesingle{}UserID\textquotesingle{}}\NormalTok{]]}\OtherTok{=}\NormalTok{user.num[X[i,}\StringTok{\textquotesingle{}UserID\textquotesingle{}}\NormalTok{]]}\SpecialCharTok{+}\DecValTok{1}
\NormalTok{    profile.rat[X[i,}\StringTok{\textquotesingle{}ProfileID\textquotesingle{}}\NormalTok{]]}\OtherTok{=}\NormalTok{profile.rat[X[i,}\StringTok{\textquotesingle{}ProfileID\textquotesingle{}}\NormalTok{]]}\SpecialCharTok{+}\NormalTok{X[i,}\StringTok{\textquotesingle{}Rating\textquotesingle{}}\NormalTok{]}
\NormalTok{    profile.num[X[i,}\StringTok{\textquotesingle{}ProfileID\textquotesingle{}}\NormalTok{]]}\OtherTok{=}\NormalTok{profile.num[X[i,}\StringTok{\textquotesingle{}ProfileID\textquotesingle{}}\NormalTok{]]}\SpecialCharTok{+}\DecValTok{1}
\NormalTok{\}}
\NormalTok{user.ave}\OtherTok{=}\NormalTok{user.rat}\SpecialCharTok{/}\NormalTok{user.num}
\NormalTok{profile.ave}\OtherTok{=}\NormalTok{profile.rat}\SpecialCharTok{/}\NormalTok{profile.num}
\NormalTok{X1}\OtherTok{=}\FunctionTok{big.matrix}\NormalTok{(}\AttributeTok{nrow=}\FunctionTok{nrow}\NormalTok{(X), }\AttributeTok{ncol=}\FunctionTok{ncol}\NormalTok{(X), }\AttributeTok{type=} \StringTok{"double"}\NormalTok{, }
              \AttributeTok{dimnames=}\FunctionTok{list}\NormalTok{(}\ConstantTok{NULL}\NormalTok{, }\FunctionTok{c}\NormalTok{(}\StringTok{\textquotesingle{}UsrAveRat\textquotesingle{}}\NormalTok{,}\StringTok{\textquotesingle{}PrfAveRat\textquotesingle{}}\NormalTok{,}\StringTok{\textquotesingle{}Rat\textquotesingle{}}\NormalTok{)))}
\NormalTok{X1[,}\StringTok{\textquotesingle{}Rat\textquotesingle{}}\NormalTok{]}\OtherTok{=}\NormalTok{X[,}\StringTok{\textquotesingle{}Rating\textquotesingle{}}\NormalTok{]}
\NormalTok{X1[,}\StringTok{\textquotesingle{}UsrAveRat\textquotesingle{}}\NormalTok{]}\OtherTok{=}\NormalTok{user.ave[X[,}\StringTok{\textquotesingle{}UserID\textquotesingle{}}\NormalTok{]]}
\NormalTok{X1[,}\StringTok{\textquotesingle{}PrfAveRat\textquotesingle{}}\NormalTok{]}\OtherTok{=}\NormalTok{profile.ave[X[,}\StringTok{\textquotesingle{}ProfileID\textquotesingle{}}\NormalTok{]]}

\CommentTok{\#Normal Method}
\NormalTok{fit }\OtherTok{\textless{}{-}} \FunctionTok{lm}\NormalTok{(Rat }\SpecialCharTok{\textasciitilde{}}\NormalTok{ UsrAveRat }\SpecialCharTok{+}\NormalTok{ PrfAveRat, }\FunctionTok{as.data.frame}\NormalTok{(}\FunctionTok{as.matrix}\NormalTok{(X1)))}

\CommentTok{\#Coefficients and R2}
\FunctionTok{summary}\NormalTok{(fit)}\SpecialCharTok{$}\NormalTok{coefficients[}\DecValTok{1}\SpecialCharTok{:}\DecValTok{3}\NormalTok{]}
\end{Highlighting}
\end{Shaded}

\begin{verbatim}
## [1] -2.1270532  0.4459886  0.9121571
\end{verbatim}

\begin{Shaded}
\begin{Highlighting}[]
\FunctionTok{summary}\NormalTok{(fit)}\SpecialCharTok{$}\NormalTok{r.squared}
\end{Highlighting}
\end{Shaded}

\begin{verbatim}
## [1] 0.6294795
\end{verbatim}

\begin{Shaded}
\begin{Highlighting}[]
\CommentTok{\#Sub{-}sampling Method (Sample 100 times of 1000 sample size)}
\NormalTok{coeff }\OtherTok{\textless{}{-}} \FunctionTok{c}\NormalTok{()}
\ControlFlowTok{for}\NormalTok{ (i }\ControlFlowTok{in} \DecValTok{1}\SpecialCharTok{:}\DecValTok{100}\NormalTok{) \{}
\NormalTok{  n1 }\OtherTok{\textless{}{-}} \FunctionTok{as.integer}\NormalTok{(}\FunctionTok{trunc}\NormalTok{(}\FunctionTok{runif}\NormalTok{(}\DecValTok{1}\NormalTok{, }\DecValTok{0}\NormalTok{, N}\DecValTok{{-}1000}\NormalTok{)))}
\NormalTok{  n2 }\OtherTok{\textless{}{-}}\NormalTok{ n1 }\SpecialCharTok{+} \DecValTok{1000}
\NormalTok{  fit }\OtherTok{\textless{}{-}} \FunctionTok{lm}\NormalTok{(Rat }\SpecialCharTok{\textasciitilde{}}\NormalTok{ UsrAveRat }\SpecialCharTok{+}\NormalTok{ PrfAveRat, }\FunctionTok{as.data.frame}\NormalTok{(}\FunctionTok{as.matrix}\NormalTok{(X1[n1}\SpecialCharTok{:}\NormalTok{n2,])))}
\NormalTok{  coeff }\OtherTok{\textless{}{-}} \FunctionTok{rbind}\NormalTok{(coeff, }\FunctionTok{summary}\NormalTok{(fit)}\SpecialCharTok{$}\NormalTok{coefficients[}\DecValTok{1}\SpecialCharTok{:}\DecValTok{3}\NormalTok{])}
\NormalTok{\}}

\CommentTok{\#Coefficients and R2}
\FunctionTok{colMeans}\NormalTok{(coeff)}
\end{Highlighting}
\end{Shaded}

\begin{verbatim}
## [1] -2.1556274  0.4454397  0.9170613
\end{verbatim}

\begin{Shaded}
\begin{Highlighting}[]
\NormalTok{preds }\OtherTok{\textless{}{-}} \FunctionTok{as.matrix}\NormalTok{(}\FunctionTok{cbind}\NormalTok{(}\DecValTok{1}\NormalTok{, X1[, }\DecValTok{1}\SpecialCharTok{:}\DecValTok{2}\NormalTok{]))}\SpecialCharTok{\%*\%}\FunctionTok{as.matrix}\NormalTok{(}\FunctionTok{colMeans}\NormalTok{(coeff))}
\NormalTok{actual }\OtherTok{\textless{}{-}} \FunctionTok{as.matrix}\NormalTok{(X1[,}\DecValTok{3}\NormalTok{])}
\NormalTok{rss }\OtherTok{\textless{}{-}} \FunctionTok{sum}\NormalTok{((preds }\SpecialCharTok{{-}}\NormalTok{ actual) }\SpecialCharTok{\^{}} \DecValTok{2}\NormalTok{)}
\NormalTok{tss }\OtherTok{\textless{}{-}} \FunctionTok{sum}\NormalTok{((actual }\SpecialCharTok{{-}} \FunctionTok{mean}\NormalTok{(actual)) }\SpecialCharTok{\^{}} \DecValTok{2}\NormalTok{)}
\NormalTok{rsq }\OtherTok{\textless{}{-}} \DecValTok{1} \SpecialCharTok{{-}}\NormalTok{ rss}\SpecialCharTok{/}\NormalTok{tss}
\NormalTok{rsq}
\end{Highlighting}
\end{Shaded}

\begin{verbatim}
## [1] 0.629465
\end{verbatim}

The sub-sampling method for big data linear regression obtained similar
coefficients values as the normal method while being less
computationally expensive.

\end{document}
